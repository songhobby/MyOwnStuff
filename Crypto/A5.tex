\documentclass[11pt]{article}
\pagestyle{empty}

\oddsidemargin -0.25in
\textwidth 7.0in
\topmargin 0.0in
\headheight 0.0in
\headsep 0.0in
\topskip 0.0in
\footskip 0.4in
\textheight 8.8in

\def\FF{{\Bbb F}}
\def\EE{{\Bbb E}}
\usepackage{amssymb}
\usepackage{algorithm}
\usepackage{algpseudocode}
\usepackage{amsmath}
\newcommand{\ZZ}{{\mathbb Z}}
\newcommand{\NN}{{\mathbb N}}
\def\DES{{\textup{DES}}}
\def\mod{{\textup{mod }}}
\def\rank{{\textup{rank}}}
\def\ord{{\textup{ord}}}
\newcommand{\cat}{\mbox{$\, \| \,$}}
\newcommand{\la}{\leftarrow}
\newcommand{\floor}[1]{\lfloor #1 \rfloor}

\begin{document}

\noindent
{\large\bf CO 487/687: Assignment \#5} \hfill Due date: April 4 (Wednesday),
11:59 pm

\hfill\hrule

\vspace*{2mm}
\noindent
{\bf NOTE}: You will be emailed a Crowdmark link for submitting the
assignment on March 23. If you do not receive the link, please
send an email to ajmeneze@uwaterloo.ca.

\begin{enumerate}

%*************************************************************************
\item {\bf Elliptic curve computations} (10 marks)\\
Consider the elliptic curve $E \; : \; Y^2 = X^3 + 10X + 16$ defined over
$\ZZ_{17}$.
\begin{enumerate}
\item Find $E(\ZZ_{17})$, the set of $\ZZ_{17}$-rational points on $E$.
\paragraph{Solution, all the '=' sign are modulo by 17}
$0^2=0, 1^2=1, 2^2=4, 3^2=9,4^2=16,5^2=8, 6^2=2, 7^2=15,8^2=13,9^2=13,10^2=15,11^2=2, 12^2=8, 13^2=16, 14^2=9, 15^2=4, 16^2=1$\\
Let $x=0, y^2= 16, y=\pm 4=4, 13$ The same for $x=1,2,3,\dots 16$
All the rational points are $\{ \infty, (0,4),(0,13),(4,1),(4,16),(5,2),(5,15) (7,2),(7,15),(8,8),(8,9),(9,6),(9,11) \}$
\item What is $\# E(\ZZ_{17})$? (Check: $\# E(\ZZ_{17})$ is prime.)
\paragraph{Solution} Its 13 and its a prime.
\item Find a generator of $E(\ZZ_{17})$.
\paragraph{Solution}
Any point except $\infty$ is a generator as the $\# E(\ZZ_{17})$ is prime.



\item Let $P = (5,2)$, $Q=(9,11)$, $R=(9,6) \in E(\ZZ_{17})$.
      Compute the following points:\\
      (i) $P+Q$.~~~ \\
      $\lambda = \frac{11-2}{9-5}=9\cdot 4^{-1}=9\cdot 13 = -2$\\
      $x=(-2)^2 - 9-5=7, y=-((-2)\cdot (7-5)+2) = 2$, Thus, $P+Q=
      (7,2)$\\
      (ii) $Q+R$.~~~ \\
      $\infty$\\
      (iii) $2R$.~~~ \\
      (8,8)\\
      (iv) $2018R$. \\
      $2018R= (2018 \bmod 13)R=3R=(4,1)$\\
\item Determine $\log_P R$.
\paragraph{Solution} $2P=(7,15),3P=(9,6)$ Thus, $\log_P R = 3$
\end{enumerate}

\newpage
%*************************************************************************
\item {\bf Point multiplication} (10 marks)\\
Let $E : Y^2 = X^3 + aX + b$ be an elliptic curve defined over $\ZZ_p$.
Let $n = \# E(\ZZ_p)$, and suppose that $n$ is prime. Design and analyze
a \emph{polynomial-time} algorithm (repeated double-and-add) which, on input
$p$, $a$, $b$, $n$, $P \in E(\ZZ_p)$ and $m \in [1,n-1]$, outputs $mP$.
\paragraph{Solution}
Write $m$ in binary representation as $m_i, i = 0,1,2,\dots, \floor{\log m}$\\
\begin{algorithm}
\caption{Double and Add}
\begin{algorithmic}[1]
\State $X \gets P, Q \gets \infty$
\For {i from 0 to $\floor{\log m}$}
\State $Q \gets m_iX+Q$ \label{1}
\State $X \gets 2X$ \label{2}
\EndFor
\State \textbf{Output} Q
\end{algorithmic}
\end{algorithm}
\paragraph{Analysis} The For loop gets executed $\floor{\log m}$ times and each loop only contains constant computation, namely line \ref{1} and \ref{2}. So the total runtime is $O(\floor{\log m} \cdot Constant) \sim O(\log m)$ which is polynomial in terms of the length of m.

%*************************************************************************
\newpage
\item {\bf Elliptic curve signature scheme} (10 marks)\\
Let $p$ be a prime, and let $E$ be an elliptic curve defined over $\ZZ_p$
with $\# E(\ZZ_p) = n$ (a prime). Let $P$ be a generator of $E(\ZZ_p)$,
and let $H$ be a cryptographic hash function. Alice
selects a private key $a \in_R [1,n-1]$, and computes her public key
$A = aP$. She signs a message $m \in \{0,1\}^*$ as follows:
\begin{itemize}
\itemsep 0mm
\item[i)] Select $k \in_R [1,n-1]$ and compute $R = kP$.
\item[ii)] Compute $e = H(m,R)$.
\item[iii)] Compute $s = (ae+k) \bmod{n}$.
\item[iv)] Alice's signature on $m$ is $(s,e)$.
\end{itemize}

\begin{enumerate}
\item Describe a reasonable procedure for verifying Alice's signature
      $(s,e)$ on a message $m$. Justify the \emph{correctness} of your
      verification algorithm. (You do not have to justify the \emph{security}
      of the signature scheme.)
      \paragraph{Solution}
      Compute $R'=sP-eA, e' = H(m, R')$. If $e'=e$ ACCEPT, otherwise REJECT.\\
      \[
      H(m, R')=H(m, sP-eA) = H(m, (s-ea)P) = H(m, kP) = H(m, R) = e
      \]
\item Suppose that Alice uses the same $k$ to sign two different messages
      $m_1$ and $m_2$. Show how an adversary who knows these messages and
      their signatures can efficiently (and with high probability) determine
      Alice's private key.
      \paragraph{Solution}
      Since $R=(s-ea)P=sP-eA$, for two message signature pairs with the same $k$, $s_1P-e_1A =R_1=kP=R_2=s_2P-e_2A$
      \[
      s_1P-e_1A =(s_1-e_1a)P = (s_2-e_2a)P=s_2P-e_2A
      \]
      We have $s_1-e_1a = s_2-e_2a$ as $P$ is a generator so k is unique.
      \[
      a = (s_1-s_2)(e_1-e_2)^{-1}
      \]
      As long as $e_1$ and $e_2$ are different (Since p is a prime, the inverse of $(e_1-e_2)$ always exists as long as they are not equal), anyone can compute the private key. Assume H is collision resistant, then with high probability $e_1\neq e_2$ as otherwise one could've found a collision $(m_1, R), (m_2, R)$.
\end{enumerate}

%*************************************************************************
\newpage
\item {\bf Elliptic curve hash function} (10 marks)\\
Let $p$ be a 256-bit prime, and let $E$ be an elliptic
curve defined over $\ZZ_p$ with $\# E(\ZZ_p)=n$ a prime.
Let $P, Q \in_R E(\ZZ_p)$ be points, neither of which is the point at
infinity. Define the function $H : [0,n-1] \times [0,n-1] \longrightarrow
E(\ZZ_p)$ by $H((a,b)) = aP+bQ$. That is, messages are
pairs $(a,b)$ of integers in the interval $[0,n-1]$, and the
hash of such a message is the elliptic curve point $aP+bQ$.
Prove, under a reasonable computational assumption, that $H$ is collision
resistant.
\paragraph{Solution} Assume DL in Elliptic Curve is computationally infeasible, H is collision-resistant. Suppose H is not collision-resistant. Then one can efficiently finds $(a_1,b_1), (a_2,b_2)$ with either $a_1 \neq a_2$ or $b_2 \neq b_2$ as otherwise the plaintext would be the same.
\[
a_1P+b_1Q = a_2P+b_2Q
\]
WLOG, $a_1\neq a_2$.
\[ P=(a_1-a_2)^{-1}(b_2-b_1)Q
\]
So one can compute $\log_P Q = (a_1-a_2)^{-1}(b_2-b_1)$


\end{enumerate}


\hfill\hrule

\newpage
\vspace*{2mm}
\noindent
You should make an effort to solve all the problems on your own.
You are also welcome to collaborate on assignments with other students
presently enrolled in CO~487/687. However, \emph{solutions must be
written up by yourself}. If you do collaborate, please \emph{acknowledge
your collaborators} in the write-up for each problem. \emph{If you
obtain a solution with help from a book, research paper, a web site,
or elsewhere, please acknowledge your source}. You are \emph{not}
permitted to solicit help from online bulletin boards, chat groups,
newsgroups, or solutions from previous offerings of the course.

\vspace*{2mm}
\noindent
The assignment should be submitted via Crowdmark before {\bf 11:59 pm
on April~4}. Late assignments will not be accepted except in
\emph{very} special circumstances (usually a documented illness of
a serious nature). A high workload because of midterm tests and assignments
in other courses will \emph{not} qualify as a special circumstance.

\vspace*{4mm}
\noindent
{\bf Instructor and TA office hours}:
\begin{tabbing}
Monday:~~~~~~~\=1:00 pm -- 2:00 pm~~~~~~~~\= Alessandra Graf (MC 5029)\\
\> 3:00 pm -- 5:30 pm \> Alfred Menezes (MC 5026)\\
Tuesday: \> 10:30 am -- 11:30 am \> Priya Soundararajan (MC 5466)\\
\> 11:30 am -- 12:30 pm \> Sam Jaques (QNC 4114)\\
\> 1:00 pm -- 2:00 pm \> Luis Ruiz-Lopez (MC 5486)\\
\> 3:00 pm -- 4:00 pm \> Elena Bakos Lang (MC 5474)\\
Thursday: \> 2:00 pm -- 3:00 pm \> Chris Leonardi (MC 5494)\\
Friday: \> 1:00 pm -- 3:00 pm \> Alfred Menezes (MC 5026)
\end{tabbing}

\hfill\hrule

\end{document}
