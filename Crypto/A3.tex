\documentclass[11pt]{article}
\pagestyle{empty}

\oddsidemargin -0.25in
\textwidth 7.0in
\topmargin 0.0in
\headheight 0.0in
\headsep 0.0in
\topskip 0.0in
\footskip 0.4in
\textheight 8.8in

\def\FF{{\Bbb F}}
\def\EE{{\Bbb E}}
\usepackage{amssymb}
\usepackage{mathtools}
\newcommand{\ZZ}{{\mathbb Z}}
\newcommand{\NN}{{\mathbb N}}
\def\DES{{\textup{DES}}}
\def\mod{{\textup{mod }}}
\def\rank{{\textup{rank}}}
\def\ord{{\textup{ord}}}
\newcommand{\cat}{\mbox{$\, \| \,$}}
\newcommand{\la}{\leftarrow}
\newcommand{\floor}[1]{\lfloor #1 \rfloor}

\usepackage{graphicx}  %for including .jpeg files

\begin{document}

\noindent
{\large\bf CO 487/687: Assignment \#3} \hfill Due date: February 28 (Wednesday),
11:00 am

\hfill\hrule

\vspace*{2mm}
\noindent
{\bf NOTE}: You will be emailed a Crowdmark link for submitting the
assignment on February 14. If you do not receive the link, please
send an email to ajmeneze@uwaterloo.ca.

\begin{enumerate}

%*************************************************************************
\item {\bf Hash functions} (10 marks)
\begin{enumerate}
\item Is a preimage-resistant hash function necessarily second-preimage
resistant? (Justify your answer.)
\paragraph{Solution} No. Suppose H is a uniform preimage-resistant hash function $H:\{0,1\}^* \to \{0,1\}^n$. Define $H':\{0,1\}^* \to \{0,1\}^n$ to be \\
\[ H'(x)=
\begin{cases}
H(x), & \text{if x is even}\\
H(x+1), & \text{if x is odd}
\end{cases}
\]\\
Given any $x\in \{0,1\}^*$, if x is even, then $x-1$ is a second-preimage as $H'(x-1)=H(x-1+1)=H(x)=H'(x)$. If x is odd, then $x+1$ is a second-preimage. So $H'$ is not second-preimaage resistant.\\
Suppose $H'$ is not preimage-resistant. Given a randomly chosen hash value $h\in_R\{0,1\}^n$ produced by $H$, with probability about 0.5 h has a preimage which is even (since h is uniform). If so we can find its preimage $x'$ by invoking the preimage finding algorithm of $H'$, namely $x'$ s.t. $H'(x')=h$. Then $H(x')=H'(x')=h$. Since this 0.5 probability is not negligible, $H$ is not preimage-resistant. By contradiction, $H'$ is preimage-resistant.

\item Suppose that $H: \{0,1\}^* \rightarrow \{0,1\}^n$ is a collision-resistant
hash function. Define a new hash function $G : \{0,1\}^* \rightarrow
\{0,1\}^n$ by $G(x) = H(H(x))$. Prove that $G$ is also collision resistant.\\
(Note: As mentioned in class, such statements are best proven using the
contrapositive statements. That is, you should prove that if $G$ is not
collision resistant, then $H$ is not collision resistant.)
\paragraph{Solution}
Suppose G is not collision resistant. There then exists a collision finding algorithm $CF$ that outputs $x,y$ s.t. $G(x)=G(y)=H(H(x))=H(H(x))$. Compute $H(x),H(y)$. If $H(x) \neq H(y)$, then $H(x), H(y)$ is a collision since $H(H(x))=H(H(y))$; Otherwise $x, y$ is a collision. So there exists such a collision finding algorithm for H as well. Therefore G is collision-ressistant if H is collision-ressistant.
\end{enumerate}


%*************************************************************************
\item {\bf Hash functions} (10 marks)\\
Let $H$ be an iterated hash function with compression function
$f : \{0,1\}^{n+r} \rightarrow \{0,1\}^n$. (See slide 155 for a complete
description of an iterated hash function.) Suppose that you are
given an algorithm $A$ which, on input $z \in \{0,1\}^n$, finds
$s,t \in \{0,1\}^r$, $s \neq t$, such that $f(z,s)=f(z,t)$.
Suppose that the running time of $A$ is $T$. Let $d$ be a positive
integer, and let $D=2^d$. Show how $A$ can be used to find, in time
approximately $d T$, $D$ pairwise distinct messages
$m_1,m_2,\ldots,m_D$ such that $H(m_1)=H(m_2)=\cdots=H(m_D)$.
\paragraph{Solution}
Invoke A to find d distinct pairs $(s_i,t_i),i=1,2,\dots,d$ such that $s_i \neq t_i, s_i \neq t_j$ if $i\neq j$ and $f(H_{i-1},s_i)=f(H_{i-1},t_i)$ where $H_{i-1}=f(H_{i-2},s_{i-1})=f(H_{i-2},t_{i-1})$ for $i > 1$ and $H_0$ is an arbitrarily chosen n-bit binary string. Let $k$ be a d-bit binary string and $k_i$ is the $i$th bit of $k$. Construct string $M_k$ by concatenating d s-bit binary strings where the $i$th block is $s_i$ if $k_i$ is 0 and $t_i$ if $k_i$ is 1. There are in total $2^d$ different messages as k varies from $0^d$ to $1^d$.
Since $f(H_0,M_{k1})=f(H_0,M_{k'1})=f(H_0,s_1)=f(H_0,t_1)$ and $f(H_{i-1},M_{ki})=f(H_{i-1},M_{k'i})=f(H_{i-1},s_i)=f(H_{i-1},t_i)$ where $k\neq k'$ and $M_{ki}$ is the $i$th block of $M_k$, $H_d=f(H_{d-1},M_{kd})$ is equal to $f(H_{d-1},M_{kd})$. The length block $b$ for both messages $M_k$ and $M_k'$ are of the same, so the final result $H(M_k) = f(H_d,b) = H(M_k')$

%*************************************************************************
\item {\bf MAC schemes derived from hash functions} (10 marks)\\
Let $f : \{0,1\}^{256} \rightarrow \{0,1\}^{128}$ be a compression function.
Let $H : \{0,1\}^* \rightarrow \{0,1\}^{128}$ be an iterated hash function
derived from $f$ as follows: To hash a message $m \in \{0,1\}^*$, do the
following:
\begin{itemize}
\item[(i)] Append a single 1 bit to the right of $m$, followed by just
      enough $0$'s so that the bitlength of the resulting message $m'$
      is a multiple of 128. Let the 128-bit blocks of $m'$ be
      $m_1, m_2, \ldots, m_{\ell}$.
\item[(ii)] Let $b$ be the 128-bit binary representation of $\ell$.
\item[(iii)] Compute $H_1 = f(m_1,0)$, $H_i = f(m_i,H_{i-1})$ for
             $2 \leq i \leq \ell$. Then $H(m) = f(b,H_\ell)$.
\end{itemize}
Consider now the following four MAC schemes each with 128-bit secret key
$k$:
\begin{enumerate}
\item $\mbox{MAC}_k(m) = H(m) \oplus k$.
\item $\mbox{MAC}_k(m) = H(k,m) \oplus k$.
\item $\mbox{MAC}_k(m) = H(k,m) \oplus H(m)$.
\item $\mbox{MAC}_k(m) = H(m,k) \oplus k$.
\end{enumerate}
Show that 3 of these 4 MAC schemes are insecure. That is, describe
attacks on 3 of these 4 MAC schemes to demonstrate that they do not resist
existential forgery by chosen-message attacks. (Note: It would help to
to make your attacks as ``practical'' as possible---that is, use as
few hash function evaluations as you can, and use as few calls to the
MACing oracle as you can.)\\
(Note: As with all cryptographic systems considered in the course, you
can assume you know all the details of the cryptographic system except
for the secret key chosen by Alice and Bob. So, in this problem, you
can assume that you know the complete description of $f$ and $H$, but
you do not know the secret key $k$.)
\paragraph{Solution}
(a) is insecure. Chose a message M and call MAC oracle once to get $c=H(m)\oplus k$. Compute $h=H(M)$ and recover the key $k=c\oplus h$. For any message m, $MAC_k(m)=H(m)\oplus k$ where k is now known.\\
(b) is insecure.
(c) is insecure. For a message m, call the MAC oracle to get $c=H(k,m)\oplus H(m)$. Then $H(k,m)=c\oplus H(m)$ Append a single 1 bit to the right of m followed by enough 0's to make the bit-length of the resulting message $m'$ multiple of 128. Let b be the 128-bit binary representation of the number of 128-bit blocks.
One can compute the hash value of $(k,m'||b||y)$ for any message $y$ (assume $y$ has bit length $128t$) as
\[
H(k,m'||b||y)=f(b+t,H_{l+1+t})\] \[ \text{where }H_{l+1+j}=f(y_j,H_{l+j}),
H_{l+1}=H(m)=f(b,H_l), H_0=f(k,0), H_1=f(m_1,H_0)
\]
Thus, one can compute the tag for this new message $m'||b||y$
\[
MAC_k(m'||b||y)=H(k,m'||b||y)\oplus H(m'||b||y)
\]

\newpage
%*************************************************************************
\item {\bf Algorithm analysis} (10 marks) \\
Describe a polynomial-time algorithm which, on input $n \in \NN$,
determines whether $n$ is an integer fifth power; that is, your algorithm
should determine if there exists an integer $a \in \NN$ such that $n
= a^5$.  ($\NN$ is the set of natural numbers.) Using big-O notation,
give an upper bound on the number of bit operations your algorithm takes.
\paragraph{Solution}
(1) down=0, up=n. Compute $mid=\floor{\frac{down+up}{2}}$
(2) if $down == up$, Output FALSE;
    else if $mid^5 > n$, up=mid;
    else if $mid^5 < n$, down=mid;
    else Output TRUE.
    \\
This algorithm takes at most $log_2n$ steps till it returns, so the running time is bounded by $\lceil log_2n \rceil+1$. For each step, to compute $mid^5$ one can compute $mid^2$, $mid^4$, $mid^5$. This requires 3 multiplications and additionally for each step the algorithm does at most 3 comparisons, one addition and one full division. So the total runnint time is (bitlength is t) $O((t+1)(3t+(3+2)t^2+t+t^2))$ which is polynomial.


%*************************************************************************
\item {\bf RSA computations} (10 marks)\\
Alice chooses $n=1073$ and $e=715$ for use in the basic RSA public-key
encryption scheme.
\begin{enumerate}
\item What is Alice's private key?
\item Encrypt the message $m=17$ for Alice.
\end{enumerate}
(The purpose of this problem is to make sure that you remember how to
perform the basic RSA operations. You may use a calculator, but please
show the main steps of your calculations. If you write a computer program,
please hand in the program.)
\paragraph{Solution}
$1073=29\times 37\\
\Phi(1073)=28\times 36=1008\\
1008=715\times 1 + 293\\
715=293\times 2+129\\
293=129\times 2 +35\\
129=35\times 3+24\\
35=24\times 1+11\\
24=11\times 2 + 2\\
11=2\times 5 + 1\\
5=1\times 5\\
547\times 715=388\times 1008+1\\
\text{Thus } 715\times 547 \equiv 1 \pmod{1008}$\\
(a)
Therefore $547\times 715\equiv 1 \pmod{1008}$, 547 is the private key.
\\
(b) $c\equiv 17^{715}\equiv 17^(2^9+2^7+2^6+2^3+2^1+1)\equiv 853 \pmod{1073}$.
For part b the python script for repeated square and multiply is as follows
\begin{verbatim}
lst="1011001011" # the binary representation of 715
mult = 17
result = 2
for st in reversed(lst):
    if (st=='1'):
        result = (result * mult) % 1073
    mult=mult**2 % 1073
print (result)
\end{verbatim}


\end{enumerate}


\hfill\hrule

\vspace*{2mm}
\noindent
You should make an effort to solve all the problems on your own.
You are also welcome to collaborate on assignments with other students
presently enrolled in CO~487/687. However, \emph{solutions must be
written up by yourself}. If you do collaborate, please \emph{acknowledge
your collaborators} in the write-up for each problem. \emph{If you
obtain a solution with help from a book, research paper, a web site,
or elsewhere, please acknowledge your source}. You are \emph{not}
permitted to solicit help from online bulletin boards, chat groups,
newsgroups, or solutions from previous offerings of the course.

\vspace*{2mm}
\noindent
The assignment should be submitted via Crowdmark before {\bf 11:00 am
on February~28}. Late assignments will not be accepted except in
\emph{very} special circumstances (usually a documented illness of
a serious nature). A high workload because of midterm tests and assignments
in other courses will \emph{not} qualify as a special circumstance.

\vspace*{4mm}
\noindent
{\bf Instructor and TA office hours}:
\begin{tabbing}
Monday:~~~~~~~\=1:00 pm -- 2:00 pm~~~~~~~~\= Alessandra Graf (MC 5029)\\
\> 3:00 pm -- 5:30 pm \> Alfred Menezes (MC 5026)\\
Tuesday: \> 10:30 am -- 11:30 am \> Priya Soundararajan (MC 5466)\\
\> 11:30 am -- 12:30 pm \> Sam Jaques (QNC 4114)\\
\> 1:00 pm -- 2:00 pm \> Luis Ruiz-Lopez (MC 5486)\\
\> 3:00 pm -- 4:00 pm \> Elena Bakos Lang (MC 5474)\\
Thursday: \> 2:00 pm -- 3:00 pm \> Chris Leonardi (MC 5494)\\
Friday: \> 1:00 pm -- 3:00 pm \> Alfred Menezes (MC 5026)
\end{tabbing}

\hfill\hrule

\end{document}
