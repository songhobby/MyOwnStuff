\documentclass[12pt]{article}
 \usepackage[margin=1in]{geometry}
\usepackage{amsmath,amsthm,amssymb,amsfonts}

\newcommand{\N}{\mathbb{N}}
\newcommand{\Z}{\mathbb{Z}}

\begin{document}

%\title{CO 487 Assignment 1}
%\author{Haobei Song (20629096)}
%\maketitle

\begin{enumerate}
\item The key letter is 's' and the key word is 'impulse'. The name of the book is 'The Annotated Anne of Green Gables'. I found the relative frequency of each english letter in literature, say it's $a_1,a_2,\dots a_{26}$. Let $b_1,b_2,\dots b_{26}$ be the relative frequency in the ciphertext corresponding to the english alphabet. Find the permutations of b's that have larger value of $L=a_1c_1,a_2c_2,\dots a_{26}c_{26}$ where $c_i$'s is a permutation of $b_i$'s. Decode the ciphertext by the 5 permutations that have the largest value of L and see which one looks more like english.
\newpage

\item
\begin{enumerate}
\item
Solve the linear equation $c=mA+b$. Since A is invertable, $m=(c-b)A^{-1}$

\item
Let $\pmb{m}=\pmb{0}$, then $\pmb{c}=\pmb{b}$. Then let $\pmb{m_i}$ be the plaintext with the $i$th entry 1 and the rest all $0$'s. Get the corresponding ciphertext $\pmb{c_i}$ and $\pmb{c_i - b}$ is the $i$th row of $\pmb{A}$

\item
Select any 3 ciphertext $c_1,c_2,c_3$ where $c_i \neq c, c_1+c_2-c_3=c$. (there are many choices to do this and it's easy) Obtain the corresponding plaintexts $m_1,m_2,m_3$ and we have
\begin{align}
m_1A+b&=c_1\\
m_2A+b&=c_2\\
m_3A+b&=c_3\\
(1)+(2)-(3) \text{ we have}& \\
(m_1+m_2-m_3)A+b&=(c_1+c_2-c_3)\\
\end{align}
Since the encryption is a one-to-one map, $m_1+m_2-m_3$ is the correspoinding plaintext for $c$.
\end{enumerate}
\newpage

\item
\begin{enumerate}
\item
(Omiting the modulo notation) Denote the original permutation $S^0_i$ . For the first byte, $i=1$, $j=2$, $S[2]=S^0[1]=2, S[1]=S^0[2]$, $t=S^0[2]+S^0[1]$.
For the second byte, $i=2$, $j=S[2]+2=4$, $S[2]=S[4], S[4]=S[2]$ so $t=S[2]+S[4]=S^0[4]+S^0[1]$. Since $S^0_i$'s is a permutation from 0 to 255, $S^0[2] \neq S^0[4]$, the first byte and the second are definitely different.
\item
In the keystream Generator, $i$ and $j$ remain the same and thus the keystream has a repeating period at least 256 bytes as $i+256 \equiv i \pmod{256}$. XOR the first 256-byte palaintext and cipertext gives the 256 byte keystream. The entire keystream is repeating pattern of these 256-byte keystream. We can XOR the ciphertext and the entire keystream to recover the whole plaintext.
\end{enumerate}
\end{enumerate}

\end{document}
